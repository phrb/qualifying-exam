\chapter{Case Studies}
\label{chap:usecases}

In this chapter we discuss our work with autotuning in different
High-Performance Computing domains. We present each study not chronologically,
but in an order that leads to the next research steps, which are presented in
Chapter \ref{chap:julia}. Each section of this chapter presents an autotuning
study in a different domain, with its own related work, search space and
autotuner implementation.

Section \ref{sec:paramSelGPU} presents our work using the OpenTuner framework
to select compiler flags for CUDA applications. Each application took seconds
to compile and run, and we were able to measure thousands of configurations per
tuning hour. We published an initial version~\cite{bruel2015autotuningGPU} of
this study in the \textit{Simpósio em Sistemas Computacionais de Alto
Desempenho} (WSCAD), and an extended version~\cite{bruel2017autotuning} in the
\textit{Concurrency and Computation: Practice and Experience} (CCPE) journal.

Section \ref{sec:FPGA} presents our work with the OpenTuner framework for
selecting configuration parameters for a tool that generates FPGA hardware from
high-level C code. The compilation and measurement for applications in this
study took minutes to complete, and we were able to measure an order of
magnitude less configurations in comparison to the study with GPU compiler
parameters. We published this work~\cite{} in the \textit{International
Conference on Reconfigurable Computing and FPGAs} (ReConFig).
\todo[inline,author=Pedro,color=cyan]{ReConFig reference pending}

Section \ref{sec:configDPE} presents our work with autotuning hardware and
software parameters for the \textit{Dot Product Engine}, an experimental analog
computer architecture. We published an initial version~\cite{} of this work in
the \textit{International Conference on Rebooting Computing} (ICRC), and we
plan to submit further developments to the \textit{International Symposium on
Computer Architecture} (ISCA).  \todo[inline,author=Pedro,color=cyan]{This
section still needs more data}

Section \ref{sec:autotuningCloud} presents our work with developing an
extension to the OpenTuner framework that enables it to perform measurements in
public distributed computing environments such as the \textit{Google Compute
Engine}. The extension follows the client-server model, and communication is
done using a protocol that we also developed. We submitted this
study~\cite{bruel2016new} to the \textit{Brazilian Symposium on Computer
Networks and Distributed Systems} (SBRC), but it was not accepted.
\todo[inline,author=Pedro,color=cyan]{Improve this paragraph, describing future
possibilities for this work, or connecting it to our new Julia system.}
