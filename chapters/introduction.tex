\chapter{Introduction}
\label{chap:introduction}

New architectures, variety of computational devices, hard to optimize.
Inflection points, exponential growth, technology change.
Look at code optimization from search perspective.
Old code, new code, can be optimized with search.

Different levels of a program's generation and usage can be optimized.
Some levels take longer to complete, expensive to measure.
Some architectures take longer to compile, expensive to measure.
Some problems take longer to solve, expensive to measure.
How to autotune complex systems such as HACC?
How to autotune programs for a supercomputer like the Titan?
How to autotune FPGA compilation?

Current approaches require multiple evaluations.
Current approaches have no support for parallel and distributed computing.
Current approaches do not target multiple levels of stack.

New approach must leverage distributed and parallel computing power.
New approach must target multiple levels of compile stack.
New approach must take care of expensive to measure cases.

A: \citet{bilmes1997optimizing}

\section{Objectives and Expected Contributions}
\label{sec:contributions}

\section{Efforts for a Reproducible and Open Research}

We believe that openness and reproducibility are fundamental to the quality of
Computer Science research. We made efforts to ensure that our research is
reproducible and open, and we believe that these efforts help our conclusions.
In this section we describe the efforts we made.

To ensure the openness of our research we uploaded the software developed
during and for research to public hosting services. We used version control
tools and made the code available under free software licenses. We also
made available the copyright-free versions of our papers, which is a common
practice in Computer Science. The research we performed is free to be used and
modified by future researchers.

To ensure the reproducibility of our research we included hardware and software
specifications in our reports and papers, and uploaded the code to generate the
data, text, presentations and data visualizations to public hosting services.
This code is also available under free-software licenses. Researchers are able
to reproduce our results if they have access to the architectures and software
we specify.

Unfortunately, not all of our research could be made open. The hardware
implementations for the CPUs, GPUs, FPGAs and most electronic components used
to produce our results are not made available by their manufacturers. Neither
is the software used to generate FPGA hardware from its Verilog specification.
To ensure reproducibility in these cases we disclosed the hardware vendor-specified
names and characteristics, and the versions of closed software we used.

With these efforts for a reproducible and open research we believe we have
provided a better foundation for our conclusions and expected contributions.

\section{Text Structure}
\label{sec:org}
