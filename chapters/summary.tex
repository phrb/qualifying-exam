\chapter{Summary}
\label{chap:final-summary}

In this document we presented our work during the first three years of the
Direct Ph.D. program (\textit{Doutorado Direto}) of the Institute of
Mathematics and Statistics of the University of São Paulo.
Chapter~\ref{chap:autonomous} presented the Algorithm Selection Problem,
discussed the general architecture of autonomous solvers and presented a
literature overview of the autotuning research topic. In
chapter~\ref{chap:autonomous}, we also discussed the implementation of the
OpenTuner framework and problems we identified in its architecture and
execution flow.

In chapter~\ref{chap:usecases} we presented four case studies on autotuning
High-Performance Computing applications for different hardware architectures.
Three of these case studies used the OpenTuner framework, and were instrumental
in identifying the problems with the framework presented in
chapter~\ref{chap:autonomous}.  We discussed interesting properties of each
case study as they were presented, from the point of view of a domain-agnostic
autotuning system.

The case study with HPE's DPE architecture is still incomplete. We are using
NODAL in this case study and we expect that the experiments we will perform
help guiding its development.  The studies on autotuning high-level synthesis
for FPGAs and on the DPE architecture are being perform in collaboration with
Sai Rahul Chalamalasetti and Dejan Milojicic, from HPE. We expect to continue
and extend this valuable collaboration in future work.

We used the knowledge about domain-agnostic autotuning and the OpenTuner
framework, acquired during the elaboration of the case studies, to start
development of NODAL, a parallel and distributed domain-agnostic autotuning
library in the Julia language. Chapter~\ref{chap:julia} presents NODAL, and our
software architecture and implementation choices. We still have not presented
real autotuning examples with NODAL, but we expect that work with autotuning
GPU compiler parameters and with the DPE architectures will be the first case
studies.

From the case studies using OpenTuner we identified that empirical autotuners
must not rely only on stochastic local search techniques or machine learning.
For autotuning cases when the time to obtain a single measurement is in the
order of hours or greater, it is not effective for an autotuner to have to wait
for measurements in order to decide a next step. This need for a developing
more informed exploration of search spaces led us to purse a collaboration in
autotuning research with french professors Arnaud Legrand, Jean-Marc Vincent
and Brice Videau, from the University of Grenoble Alpes.

The professors have investigated the application of Design of
Experiments~\cite{fisher1937design} to other autotuning use cases involving
GPUs, and have obtained encouraging methods. They were able to systematically
find near-optimal solutions within a reasonable amount of configuration
evaluations. They are interested in extending this approach to FPGAs using real
applications and industry partners who can directly benefit from the process.

We expect that devising new robust and cheap experimental designs that enable
mixing binary, factorial and continuous variables will present an interesting
challenge.  This collaboration is therefore an interesting opportunity, and
Pedro will spend 18 months in Grenoble with funding from the CAPES/COFECUB
project, starting in November 2017. Pedro was already accepted at the Doctoral
School of the Univesity of Grenoble Alpes and will pursue a double diploma.

\section{Schedule}
\label{sec:schedule}

Table \ref{tab:sched} presents our tentative schedule for the 18 months of
co-tutelage at the University of Grenoble Alpes, in the period from 01/11/2017
to 01/05/2019. In the first semester we will develop a deeper understanding of
autotuning techniques in the context of design-of-experiments literature, as
well as begin to devise an autotuning model for expensive-to-evaluate
functions. In the subsequent semesters we will work on implementing, testing
and validating our autotuning approach. In the third semester we will write
research papers summarizing our findings and work on the Ph.D. thesis.

\begin{table}[htpb]
\centering
    \begin{tabular}{@{}p{0.36\textwidth}p{0.14\textwidth}p{0.14\textwidth}p{0.14\textwidth}@{}}
        \toprule
        \multirow{2}{*}{\textbf{Planned Research Activities}} & \multicolumn{3}{c}{\textbf{Semesters}} \\
        & \multicolumn{1}{c}{\footnotesize{11/17-05/18}} & \multicolumn{1}{c}{\footnotesize{05/18-11/18}} & \multicolumn{1}{c}{\footnotesize{11/18-05/19}} \\ \midrule
        \textit{Bibliographic Review} & \cellcolor[HTML]{C0C0C0} &  &  \\
        \addlinespace
        \textit{Autotuning Model Development} & \cellcolor[HTML]{C0C0C0} & \cellcolor[HTML]{C0C0C0} &  \\
        \addlinespace
        \textit{Implementation \& Testing} & \cellcolor[HTML]{C0C0C0} & \cellcolor[HTML]{C0C0C0} & \cellcolor[HTML]{C0C0C0} \\
        \addlinespace
        \textit{Experiments \& Validation} &  & \cellcolor[HTML]{C0C0C0} & \cellcolor[HTML]{C0C0C0} \\
        \addlinespace
        \textit{Paper \& Thesis Writing} &  &  & \cellcolor[HTML]{C0C0C0} \\ \bottomrule
    \end{tabular}
    \caption{Research Activities Schedule}
    \label{tab:sched}
\end{table}
