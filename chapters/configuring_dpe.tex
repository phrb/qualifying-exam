\section{Configuring Hardware and Software for the Dot Product Engine}
\label{sec:configDPE}

Advances in computing are often represented by fixed, application specific
hardware. For example, one of the first general purpose computers, The
Electronic Numerical Integrator and Computer (ENIAC), required re-wiring of a
plug board to program it for each application \cite{goldstine1946electronic}.
Likewise, Systolic, Wavefront~\cite{kung1984supercomputing}, and
Dataflow~\cite{iannucci1988toward} architectures each needed to be reconfigured
for a specific computation, such as matrix multiplication or convolution.  Even
Field Programmable Gate Arrays, though reconfigurable, need sophisticated
operating system (OS) support to act as anything more than a fixed system
during operation~\cite{so2006improving}.  Throughout the evolution of these
computing solutions, OS support has been very scarce, relegating these devices
to act as early accelerators and not as general purpose computers.  Even
sophisticated co-processing devices, such as Graphics Processing Units (GPU),
have highly specialized instruction sets that render their use as general
purpose computing resources difficult.

Conversely, OS designers have not easily adapted to the changes in processing
paradigms and increasing importance of power,
largely building systems that rely on a task-based workload model. Although
there have been many notable attempts at building OSes for accelerators, none
of these have been completely successful~\cite{laplante2016rethinking}.

In this study we explore the OS functionality applied to generalizing a
memristor-based accelerator, using a Dot Product Engine (DPE) as an example
that represents an analog form of computer with digital components embedded
into the architecture.  We explore generalizing this accelerator by making it
more reconfigurable and partitionable, as well as more secure.  We also
identify some of the unique characteristics of this kind of system, such as
different precision settings of DPE and trustworthiness of the deployed
configuration into DPE.

\subsection{Results}
\label{subsec:DPEres}

\subsection{Summary and Future Work}
\label{subsec:DPEconcl}
