\section{Autotuning}
\label{chap:autotuning}

Rice's conceptual framework~\cite{rice1976algorithm} formed the foundation
of autotuners in various problem domains.  In 1997, the PHiPAC
system~\cite{bilmes1997optimizing} used code generators and search scripts to
automatically generate high performance code
for matrix multiplication. Since then, systems tackled different domains with a
diversity of strategies. Whaley \emph{et al.}~\cite{dongarra1998automatically}
introduced the ATLAS project, that optimizes dense matrix multiplication
routines. The OSKI~\cite{vuduc2005oski} library provides automatically tuned
kernels for sparse matrices. The FFTW~\cite{frigo1998fftw} library provides
tuned C subroutines for computing the Discrete Fourier Transform.
Periscope~\cite{gerndt2010automatic} is a distributed online autotuner for
parallel systems and single-node performance.  In an effort to provide a common
representation of multiple parallel programming models, the INSIEME compiler
project~\cite{jordan2012multi} implements abstractions for OpenMP, MPI and
OpenCL, and generates optimized parallel code for heterogeneous multi-core
architectures.

Some systems provide generic tools that enable the implementation of
autotuners in various domains. PetaBricks~\cite{ansel2009petabricks} is a
language, compiler and autotuner that introduces abstractions, such as the
\texttt{\footnotesize either...or} construct, that enable programmers to define
multiple algorithms for the same problem.  The ParamILS
framework~\cite{hutter2009paramils} applies stochastic local search methods
for algorithm configuration and parameter tuning. The OpenTuner
framework~\cite{ansel2014opentuner} provides ensembles of techniques that
search spaces of program configurations. Bosboom \emph{et al.} and Eliahu use
OpenTuner to implement a domain specific language for data-flow
programming~\cite{bosboom2014streamjit} and a framework for recursive parallel
algorithm optimization~\cite{eliahu2015frpa}.

\subsection{OpenTuner}
\label{sec:opentuner}
\todo[inline,author=Pedro,color=cyan]{Discuss OpenTuner's limitations and shortcomings.}

\subsubsection{Context}
\label{sec:context}

\subsubsection{Software Architecture}
\label{sec:arch}

\subsubsection{Parallel and Distributed Programming in OpenTuner}
\label{sec:opentuner-parallel}
\todo[inline,author=Pedro,color=cyan]{Discuss the difficulties and
explain why they motivated the new Julia code.}

\subsection{Search Techniques}
\label{sec:techniques}

\subsubsection{Numerical Methods}
\label{subsec:num}

\subsubsection{Evolutionary Computation}
\label{subsec:tuninevolcomp}

\subsubsection{Stochastic Local Search}
\label{subsec:tuningsls}

\subsubsection{Machine Learning}
\label{subsec:tuningml}

\subsection{Benchmarks}
\label{sec:benchmarks}

\subsubsection{Solvers of NP-Complete Problems}
\label{subsec:np}

\subsubsection{Algorithm Selection and Configuration}
\label{subsec:algsel}

\subsubsection{Compiler Configuration}
\label{subsec:compilerconfig}

\subsubsection{Measurement Time}
\label{subsec:measure}
